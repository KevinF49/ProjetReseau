\documentclass[11pt]{article}

\usepackage[latin1]{inputenc}
\usepackage[T1]{fontenc}      
\usepackage[francais]{babel}  
\usepackage{amsthm}
\usepackage{amssymb}
\usepackage{amsmath}
\usepackage{mathrsfs}
\usepackage{epstopdf}
\usepackage{underscore}
\usepackage{fancyhdr}
\pagestyle{fancy}
\usepackage{geometry}
\geometry{top=2cm, bottom=2cm}
\DeclareMathOperator{\e}{e}

\date{}

\begin{document}
%\maketitle
%\section{}
%\subsection{}

\fancyhf{}
\chead{Projet R�seau}
\cfoot{\today}
\rfoot{\thepage/3}
\renewcommand{\footrulewidth}{1pt}
\thispagestyle{empty}

\begin{center}
	{\Large Cyberd�fense 2\\~\\
	ENSIBS Vannes\\~\\
	2015-2016}
\end{center}
\vspace{5cm}

\begin{center}
	{\huge \textbf{ Overview }}
	{\huge \\~\\~ SIP : Interception et chiffrement des communications.}
	{\huge
	 \\~\\ \textit{Bases r�seaux}}
\end{center}
\vspace{8cm}

\begin{center}
   {\large R�alis� par : 
   Valentin ALLAIRE,
   Dylan COIC,
   Guillaume COUCHARD,
   K�vin FAUVE}
\end{center}

\begin{center}
	{\large Encadr� par : Ma�l AUZIAS}
\end{center}
\newpage
 
%\begin{center}
%\LARGE{\textbf{\textsc{\space \space \space \space \space \space 1. Projet r�seau}}}\newline

%\Large{\textbf{\textsc{ \space \space \space \space \space \space 1.1. Overview}}}\newline

%\large{\textbf{  SIP : Interception et chiffrement des communications. }} \newline %A revoir

%\end{center}

%\textit{ \textbf{ Noms des participants  : }} \newline

%Valentin ALLAIRE, Dylan COIC, Guillaume COUCHARD, K�vin FAUVE.\newline

\textit{\textbf{ Objectifs du projet :} }\newline

L'objectif global de ce projet est tout d'abord d'intercepter une communication t�l�phonique entre deux t�l�phones IP afin de montrer par la suite qu'il est pr�f�rable de chiffrer ces communications. \newline

Plus pr�cis�ment, les objectifs sont les suivants en sachant que nous mettons une liste exhaustive : \newline

- Intercepter une communication non chiffr�e entre deux t�l�phones (ce seront des softphones dans notre cas). \newline

- Chiffrer cette communication pour montrer que cela r�soud le probl�me (si le chiffrement ne pr�sente pas de failles ...).\newline

- Monter un multi-sites et communiquer entre les deux. \newline

- Faire une configuration multi-sites en bridge et l'autre en VPN. \newline

- Montrer que le VPN est efficace en interceptant la communication entre les deux sites lorsque c'est une configuration bridge. \newline

- Intercepter une visioconf�rence. \newline

- Montrer qu'il est possible de passer outre un chiffrement TLS/SSL si celui-ci pr�sente une faille (Mise en place d'anciennes versions du protocole et exploitation des diff�rentes failles). \newline
	

\textit{ \textbf{ Connaissances que l'on d�sire acqu�rir }}\newline

Gr�ce � ce projet nous voulons savoir :\newline

- Approfondir / D�couvrir les diff�rents protocoles (SIP, TLS/SSL...). 	\newline

- Configurer un serveur asterisk. \newline

- Intercepter une communication SIP. \newline

- Chiffrer une communication. \newline

- Monter 2 sites diff�rents et les faire communiquer entre eux pour repr�senter une situation plus r�elle. \newline

- Configurer un tunnel VPN. \newline

- Finalement, nous voulons aussi apprendre � monter une infrastructure r�seau s'approchant d'une infrastructure professionnelle. \newline
\newline


\textit{\textbf{Organisation pour se partager le code}} \newline

Nous allons principalement utiliser Github pour se partager nos configurations car cela permet de faire du versionning ce qui sera tr�s utile.

Mais nous allons certainement utiliser parfois le mail pour s'envoyer certaines versions mais aussi bien s�r pour communiquer.

\end{document}